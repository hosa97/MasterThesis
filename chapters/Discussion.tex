\chapter{考察}\label{DC}

協調プログラミングの2つの要件(collective contributionとproductive interaction)の観点で考察を行う.2つの要件の達成度合いから本研究の限界を述べる.


\section{collective contribution}

成績に応じてソースコードの提供による貢献やテストによる貢献が行われていることから,各メンバが実力に見合った貢献ができていると考えられる.グループの2番手にあたるメンバはソースコードのレベルで貢献をし,3番手にあたるメンバはテストで貢献をしている.

プログラミングが苦手なメンバに対し,テストという点での貢献を促進できたことも1つの成果である.従来,3番手にあたるメンバは,プログラミングが苦手であるゆえに素材の作成を任されることが多かった.素材の作成による貢献もプロジェクトにおいて重要な作業である.テストもまた重要な作業である.テストにおいて,3番手にあたるメンバがプロジェクトに貢献する機会を用意することができた.

以上の成果は,独立同期モデルにおける取り込みの容易さに起因していると考えられる.全部取込と部分取込という2種類の取り込み方法を目的に応じて使い分けていることがわかった.


\section{productive interaction}

フォロワがドライバのソースコード編集の様子を観察していることや,積極的に取り込み操作が行われていることから,各メンバの能力が相乗的にプロジェクトに反映されている可能性がある.

観察とアンケートの自由記述からは,知識の伝播や創造的な議論が行われていることがわかった.

我々は,ソースコードに手をつけていないフォロワに,ドライバのソースコード編集の様子を観察・取り込みする機会を与えることができたことが大きな成果であると考える.非利用群においては,素材での貢献をするフォロワは,完成した素材をUSBメモリやメッセンジャーツール等でドライバに一方的に送るということが多く見られた.観察中に発見することはできなかったが,例えば画像のサイズがプログラムの動作にあっていない時に,フォロワが自らプログラムを書き換えて画像のサイズを調整するということも考えられる.このような知識創造の機会を与えることができた.

以上の成果は,独立同期モデルにおけるグループメンバの最新バージョンをリアルタイムに閲覧できることと,取り込みの容易さに起因していると考えられる.


\section{本研究の限界}

独立同期モデルが協調プログラミングの支援に一定の成果を挙げたが,いかなるグループに対しても有効に作用するわけではない.アンケートの自由記述において,グループでプログラムを組むことに対してネガティブな意見が見られた.典型的な例は,「プログラミングが苦手であるため貢献できなかった」という意見と,「個人で組むよりも気を遣ってしまい苦労をする」という意見である.

独立同期モデルが有効に働くか否かは,作品の内容,グループ内の能力差や各メンバの性格という様々な要因が重なることによって決まると考えられる.現状では,これらを測定する指標が曖昧なため,分析が困難である.例えば,プログラミング能力の指標として講義の成績を用いているが,成績がプログラミング能力に繋がるとは限らない.

支援可能なグループの幅を広げるために,協調プログラミングの実態をより明白にする必要がある.インタラクション図という形でファイルのやりとりを可視化したことで,初学者の協調プログラミングの実態を部分的に明らかにすることができた.協調プログラミング中のコミュニケーションなどのデータも取得し,インタラクション図と組み合わせたより詳細な分析が今後の課題である.