\chapter{協調プログラミング}\label{Def}


%----------2.1--------
\section{ソフトウェア開発における協調作業のモデル}

プログラミング/ソフトウェア開発における協調作業のモデルは,ソフトウェア工学の分野で30年以上にわたって議論され続けてきた.中でも,ウォータフォールモデルとアジャイルモデルの2つのモデルが主要である.ウォータフォールモデルは要件定義,デザイン,実装,テストのようなプロセスを完全に分担するモデルであり,30年以上にわたって最も有力なモデルであった.対照的にここ10年では,アジャイルモデルの人気が上昇している.アジャイルモデルは,柔軟で創造的なソフトウェア開発が要求される産業的な開発においても,状況変化や新しいソフトウェア開発の進化に適している.特に,Scrum\cite{schwaber2002gile}はソフトウェア開発方法論の中で現在最も人気のある方法である.Scrumは知識創造理論\cite{takeuchi1986new}がベースとなっている.従って,ウォーターフォールモデルとは正反対の協調作業が行われる.例えば,プロジェクトマネージャによる管理に代わって,自己組織化プロセスが推奨されることが挙げられる.Scrumでは,役割分担をするのではなく,作業を共有し暗黙知を共有することが推奨される.

現在,プロフェッショナルと教育現場の双方で,協調プログラミングでの各モデルの利点が議論されている.我々のモデルは主にアジャイルモデルと知識創造理論に基づいている.近年,参加者間の論理的なインタラクションと役割やリーダの頻繁な変更が,オープンソースソフトウェア開発コミュニティのような創造的なコミュニティで観察された\cite{kidane2007correlating}.教育現場でも同様の現象が観察されている\cite{knutas2013communication}.その他の研究においても,協調プログラミングにおいて学習者間のインタラクションの重要性が主張されている.平井らは,プログラミング学習について参加者が意見交換,競合,交渉,合意形成等を繰り返し,グループの合意としての成果を出すことを協調プログラミング学習と定義している\cite{平井佑樹2012}.

ソフトウェア開発の現場の観点から,ペアプログラミングとその利点について議論され続けている.ペアプログラミングはアジャイルモデルの前身であるeXtreme Programming(XP)\cite{beck2000extreme}の12のベストプラクティスのうちの1つである.Cockburnらは協調プログラミングのモデルとして,ペアプログラミングを拡張した「side by sideプログラミング」を提案した\cite{cockburn2000costs}.Goldmanらは,リアルタイム共同コーディング環境の使い方として,(1)授業でのプログラミング,(2)テスト駆動ペアプログラミング,(3)マイクロアウトソーシング,の3つを提案した\cite{goldman2011real}.しかし,協調プログラミングの部分的な成功にとどまり,結果は,教育での小規模のチームにおいて協調学習の促進に成功したのみであった.


%----------2.2--------
\section{入門環境における協調プログラミングの現状}\label{State}

我々は,プログラミング入門授業の場で,最終課題として1チーム2, 3名から構成されるグループプログラミングを実施してきた.現状調査のために,2013年度の授業でアンケートと観察による協調プログラミングの調査を行った.約100名の受講者に対するアンケートを行い,70件の有効回答を得た.

%半分以上を%にかえて,比較できるようにする.このままだと,その他のメンバも結構いけてる雰囲気がしてしまう.
我々が最も注目したのは,グループメンバ間の能力差についてである.まず,各グループのメンバについて次の2つの役割に分類した.

\begin{description}
	\item[ドライバ] グループ内で,コーディングにおいてプロジェクトに最も主体的に参加しているメンバ
	\item[フォロワ] ドライバ以外のメンバ
\end{description}

ドライバとフォロワの実際の記述量に関するアンケート\footnote{学習者の主観による}の結果,グループ内で最もプログラミングに対する得意意識が高い学習者は,成果物のソースコードの記述量の平均が67\%であることことがわかった.その他のメンバのソースコード記述量は,平均28\%であり,全く記述していない学習者も存在した.

%入門環境における協調プログラミングにおいて以上のような現象が発生しており,各グループのメンバについて次の2群に分類可能であることがわかった.我々は,それぞれをドライバとフォロワと名付けた.

観察結果から,2つの典型的な問題があることもわかった.

1つ目の問題は,個々の能力に見合わない不適切な役割分担である.我々は,プロジェクトに費やす時間やソースコードの記述量が,グループメンバ同士で等しくなるように課題に取り掛かることは意図していない.しかし,学習者は各グループメンバの責任を平等にするために,作業量が等しくなるように無理に役割分担を行うことがある.

2つ目の問題は,プロジェクトが完全に独立した作業から成ることである.例えば3人グループのプロジェクトにおいて,メンバそれぞれが1つのミニゲームを作り,それをまとめて3つのミニゲームから成るソフトウェアを成果物とするグループがある.この作品は,単に3つのプロジェクトを個別に開発したのみであり,協調的な作業とは言えない.
%「加算的」はこの場合の意図したいことを意味しない.


%----------2.3--------
\section{協調プログラミングの定義}

%超高度,文献でサポート出来ると良い
%「創発」,「協調学習」,「協調」の定義とかね.
我々が目指す協調プログラミングは,\ref{State}節で述べた2つの問題を解決し,グループメンバ同士のインタラクションによって,創発的に創造的な成果を生み出すような活動が行われるものである.本研究における協調プログラミングは,以下の2つの要件を満たすものとする.


\begin{description}
	\item[collective contribution] 能力が一律ではないグループのメンバ個々人が実力に見合った貢献ができていること.
	\item[productive interaction] グループメンバのインタラクションによって,各メンバの知識・能力が相乗的にプロジェクトに反映されていること.
\end{description}


collective contributionの達成は,フォロワの遠慮を取り除くことで可能であると考えられる.フォロワの遠慮は,ドライバに最新のプロジェクトを送信するように要求する時や,ドライバのソースコードを編集する時に発生することがある.フォロワの遠慮を取り除くことで,フォロワが積極的にプロジェクトに参加可能となる.結果,能力に見合った貢献が可能となると考えられる.

productive interactionの達成は,フォロワの貢献度合いに依存するものと考えられる.ドライバよりもプログラミングが苦手とされるフォロワの貢献は,ドライバや他のフォロワとのインタラクションから発生すると考えられる.