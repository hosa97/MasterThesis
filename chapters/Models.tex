\chapter{独立同期モデルの提案}\label{Model}

本章では,既存の分散型構成管理システムをモデル化し,その特徴を述べた後,本研究で初学者が利用出来るように考案したモデルを提示する.


%----------4.1---------
\section{SCMモデル}

\begin{figure}[h]
	\centering
	\includegraphics[width=0.75\linewidth]{img/SCMModel.eps}
	\caption{SCMモデル}
	\label{fig:SCMModel}
\end{figure}

SCMモデルは,gitのようなソースコードの変更履歴を記録・追跡するための分散型構成管理ツールのモデルである.SCMモデルを\figref{fig:Models}に示す.図のフォルダアイコンはプロジェクトフォルダを表す.実線矢印は同期,破線矢印は取り込みを表す(他2つのモデルも同様である).SCMモデルは以下の特徴を持つ.

\begin{enumerate}
	\item 1人で複数のブランチを管理することができる.
	\item push・fetch操作により最新バージョンを共有する.
	\item 取り込みはmarge(差分取込)を用いる.
\end{enumerate}

SCMモデルには2つの問題点がある.

1つ目は,pushやfetch,mergeといった様々な専門用語の概念や操作方法が初心者にとって複雑な点である.プログラミング入門教育においては,他に重要な学習項目が多数あるため,このような不必要な学習負荷は避けるべきである.

2つ目は,ドライバが最新のプロジェクトをpushしなければ,フォロワがドライバの最新のプロジェクトを入手することができない点である.このような状況の際に,フォロワはドライバに最新のプロジェクトをpushするように要求する必要がある.しかし,フォロワはドライバの作業を阻害してしまうことを恐れて要求できないことがある.


%----------4.2---------
\section{リアルタイム共有モデル}

\begin{figure}[h]
	\centering
	\includegraphics[width=0.75\linewidth]{img/RAEModel.eps}
	\caption{リアルタイム共有モデル}
	\label{fig:RAEModel}
\end{figure}

リアルタイム共有モデルは,Sarosのようなリアルタイムにファイルを同期し,多人数で同時に同一ファイルへの書き込みが可能なツールのモデルである.リアルタイム共有モデルを\figref{fig:Models}に示す.リアルタイム共有モデルは以下の特徴を持つ.

\begin{enumerate}
	\item グループ内の全てのメンバが1つのプロジェクトを共有する.
	\item 共有されたプロジェクトはリアルタイムに更新・表示される.
	\item 全てのメンバが共有されたプロジェクトを同時に編集できる.
\end{enumerate}

リアルタイム共有モデルの問題点は,ソースコードを編集することがフォロワにとって困難である点である.このモデルは,個人の編集がグループで共有しているファイルに直接変更を加えるため,ドライバが制作したプログラムをフォロワが壊してしまうということが起こり得る.従って,ソースコードの編集には注意が必要となり,フォロワがプロジェクトに参加しづらい.


%----------4.3---------
\section{独立同期モデル}

\begin{figure}[h]
	\centering
	\includegraphics[width=0.75\linewidth]{img/IARModel.eps}
	\caption{独立同期モデル}
	\label{fig:IARModel}
\end{figure}

独立同期モデルは,プログラミング初心者の協調プログラミングを促進するために,我々が提案するモデルである.独立同期モデルを\figref{fig:Models}に示す.独立同期モデルは以下の特長を持つ.

\begin{enumerate}
	\item 独立した個々のブランチを管理する.
	\item グループメンバの最新バージョンをリアルタイムに閲覧できる.
	\item グループメンバのソースコードを単純操作で取り込むことができる.
\end{enumerate}

特長(1)によって,複数のブランチを管理するようなSCMモデルの複雑さが解消される.グループのプロジェクトに直接干渉してしまうというリアルタイム共有モデルの問題点も解消されると考えられる.この特徴によって,フォロワはドライバのプロジェクトを直接編集してしまうような心配がなくなり,積極的にプロジェクトに貢献ができるようになると考えられる.

特長(2)はリアルタイム共有モデルの利点を採用し,SCMモデルの問題点を解決している.グループメンバに最新バージョンのpush要求を行う必要がなく,他のメンバの作業を中断する心配を解消することができる.この特長によって,フォロワはドライバへのファイル送信要求をする必要がなくなる.結果,フォロワの遠慮が減少することが予想できる.

特長(3)によって,ツールの操作方法などを学ぶ学習負荷を解消することができる.取り込みが容易であれば取り込み回数も増え,他のメンバのソースコードの閲覧・編集が行われる可能性がある.これは,productive interactionの手助けになると考えられる.