\chapter{結論}\label{CC}

プログラミング入門教育の場で,グループでプログラムを組む機会が設けられていることがある.初学者にとって協調プログラミングを行うことが困難となっている問題を解決するために,独立同期モデルを提案し,モデルに基づくシステムCheCoProを開発した.

CheCoProをプログラミング入門講義に導入し,協調プログラミングを実施した.CheCoProの利用ログとアンケートの結果を分析し,独立同期モデルの有効性の検証と協調プログラミングの実態解明を試みた.

その結果,協調プログラミングの2つの要件を充足すると解釈されるインタラクションパターンが見られた.アンケート結果も,グループメンバ間の能力差にかかわらず,個々のメンバが遠慮せず,より満足度の高い貢献を行うことができたことを示した.