\chapter{序論}
ICTを活用しながら学習者同士が互いに知識・経験・技能を豊かにし,問題解決の方法を見出し,協調的なプロセスを通して新しい知識を創造する能力が21世紀に必要な能力とされている\cite{griffin2012assessment}.ソフトウェア開発の現場においても,近年では協調的で創造的なソフトウェア開発手法のニーズが高まっている.そのような背景でアジャイルソフトウェア開発手法,特に,知識創造理論\cite{takeuchi1986new}をベースとしたScrum\cite{schwaber2002gile}が注目を集めており,教育現場でもその導入が試みられ始めている\cite{anslow2015experience}.

プログラミング入門教育においても,グループでプログラムを作成する課題が試みられてきた\cite{松浦佐江子2003}\cite{玉田春昭}.しかしながら,実際の現場では,初学者がグループプログラミングを通して,協調的な創造活動を行うことは困難である.我々の現場の観察では,典型例として,(1)プログラムの大半をグループで最も得意な学生が記述してしまい,その他のメンバはプログラムを書かない,(2)グループメンバが完全にタスクを分割して作業をしてしまう(例えば,複数の小規模なサブゲームから成るゲームをそれぞれ完全に独立して制作する)などという問題が観察される(\ref{State}節にて詳説).

初学者がグループプログラミングで使用するツールにも問題がある.理想としては,構成管理ツール(e.g. subversion, git,mercurial)を駆使してプロジェクトを進めることが望ましい.しかしながら,プロフェッショナル向けにデザインされているためツールの操作や概念が複雑で,プログラミング入門講義の学習者が使用するのは困難である.プログラミング入門教育で学習者が集中すべきなのはアルゴリズム構築であり,このようなツールの利用方法に関して学ぶ時間は用意されていない.

本研究では,これらの問題を解決するために,初学者向けの協調プログラミング支援システム「CheCoPro」を提案する.初学者が容易に利用でき,かつ能力差があるグループでも,個々人が並行的に作業し,貢献できる構成管理ツールのモデルを考案し,実際の授業で試行した.システムの利用ログを利用することで初学者のグループプログラミングにおけるインタラクションを視覚化することで,インタラクションの実態解明と合わせてツールの評価を行った.

本論文は全\ref{CC}章からなる.プログラミング教育における「協調プログラミング」の定義を\ref{Def}章で行う.\ref{RW}章で,グループプログラミングを支援する既存ツールのレビューを行う.\ref{Model}章では,初学者に協調プログラミングを支援するツールのモデル「独立同期モデル」を提案する.\ref{CH}章では,\ref{Model}章で提案したモデルに基づいて設計・実装したシステムを説明する.\ref{EM}章で評価実験の方法,\ref{RS}章ではその結果を報告する.\ref{DC}章で考察を行う.\ref{CC}章はまとめである.