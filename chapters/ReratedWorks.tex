\chapter{先行研究}\label{RW}

CSCW(Computer Supported Cooperative Work)やCSCL(Computer Supported Collaborative Learning)という分野で,グループでのプログラミングを支援するために多くのツールが提案されてきた.

Collabodeは教育目的でデザインされた最新のツールである \cite{goldman2011real}.Collabodeは,複数の学生が同時にソースコードを編集可能なリアルタイム同期環境である.Vandeventerらも同様なツールを提案している \cite{vandeventer2012codewave}.技術的には,一般的な文書のための同時編集環境のライブラリであるEtherPadを基にしている.このライブラリはGoogle Docにも用いられている.同様なツールとしてはSarosがある \cite{salinger2010saros}.SarosはEcipseのプラグインとして実装されている.このツールは指定したメンバの視野(スクロールやファイル選択)を追従する機能が実装されている.Salingerらはグループメンバの意識共有を支援し,コードレビューや遠隔でのペア・パーティプログラミングの際に有用であると主張している.

しかし,リアルタイム同期モデルは営利的なソフトウェア開発やオープンソースプロジェクトのようなプロフェッショナル向けの協調プログラミングではほとんど使われていない.開発現場では,プロフェッショナルはCVSやsubversion,GitのようなSCM(Source Configuration Management)支援ツールを30年以上使っている.現在では,Git / Githubがオープンソースコミュニティでのスタンダードなツールである.これらのツールは最新のツールにも関わらず,単にブランチ&マージモデルを支援しているだけであり,リアルタイムシェアリングはできない.

現在,教育目的でのリアルタイム同期モデルとブランチ&マージモデルの利点は明らかになっていない.一般的な文書の記述において,リアルタイム同期環境の利点の解明に取り組んでいる研究は存在する \cite{brodahl2011collaborative}\cite{zhou2012google}.Andr\'eらは,規則的な作業においては利点を有するが,複雑な創造的なタスクを同時に取り掛かることには不利であると主張している \cite{andre2014effects}.一方で,プログラミングの入門教育において,SCMを取り入れて成功したという報告はない.我々は,リアルタイム同期モデルでは学生の編集が直接的にグループの作品に影響を与えてしまうことを問題と考えている.その他の問題として,時間を共有する必要があるため授業時間外の共同作業を促進できない点が挙げられる \cite{zhou2012google}.対照的に,ブランチ&マージモデルは,遠慮や時間を共有することがないという点で融通がきく.このモデルの問題は,モデルの概念やツールの操作を学ぶために大きな認知的負荷を要することである.特に,一度コンフリクトが発生するとプロフェッショナルでさえも解決に手間がかかるほどである.

以上のように,コミュニケーションや遠隔でのペアプログラミング支援,プロフェッショナル向けのSCMが提案されている.しかし,初学者が使用可能かつ「協調プログラミング」を支援可能なシステムは開発されていない.