\documentclass[11pt, a4j, openany, oneside, dvipdfmx]{jbook}

%必要に応じて適宜
\usepackage[dvips]{graphicx}
\usepackage{latexsym}
\usepackage{amssymb}
\usepackage[hang,small,bf]{caption}
\usepackage{pdfpages} % for insert pdf

%修論用自作パッケージ
\usepackage{SLabMasterThesisStyle}

%表紙出力のための著者情報
%「修士論文」と自動で出るので,学部生はSLabMasterThesisStyle.styを編集して調整する.
\year{27} 
\title{修論作成のためのテンプレート} 
\advisingteacher{情報 太郎}
\studentid{7000-0000}
\author{静大 花子}
\department{静岡大学大学院情報学研究科 \\ 情報学専攻}


\begin{document}

%表紙出力
\maketitle

%目次出力
\tableofcontents

%ここから本文追加
%こんな感じで章ごとにtexファイルを分けると編集しやすそう
\chapter{序論}
ICTを活用しながら学習者同士が互いに知識・経験・技能を豊かにし,問題解決の方法を見出し,協調的なプロセスを通して新しい知識を創造する能力が21世紀に必要な能力とされている\cite{griffin2012assessment}.ソフトウェア開発の現場においても,近年では協調的で創造的なソフトウェア開発手法のニーズが高まっている.そのような背景でアジャイルソフトウェア開発手法,特に,知識創造理論\cite{takeuchi1986new}をベースとしたScrum\cite{schwaber2002gile}が注目を集めており,教育現場でもその導入が試みられ始めている\cite{anslow2015experience}.

プログラミング入門教育においても,グループでプログラムを作成する課題が試みられてきた\cite{松浦佐江子2003}\cite{玉田春昭}.しかしながら,実際の現場では,初学者がグループプログラミングを通して,協調的な創造活動を行うことは困難である.我々の現場の観察では,典型例として,(1)プログラムの大半をグループで最も得意な学生が記述してしまい,その他のメンバはプログラムを書かない,(2)グループメンバが完全にタスクを分割して作業をしてしまう(例えば,複数の小規模なサブゲームから成るゲームをそれぞれ完全に独立して制作する)などという問題が観察される(\ref{State}節にて詳説).

初学者がグループプログラミングで使用するツールにも問題がある.理想としては,構成管理ツール(e.g. subversion, git,mercurial)を駆使してプロジェクトを進めることが望ましい.しかしながら,プロフェッショナル向けにデザインされているためツールの操作や概念が複雑で,プログラミング入門講義の学習者が使用するのは困難である.プログラミング入門教育で学習者が集中すべきなのはアルゴリズム構築であり,このようなツールの利用方法に関して学ぶ時間は用意されていない.

本研究では,これらの問題を解決するために,初学者向けの協調プログラミング支援システム「CheCoPro」を提案する.初学者が容易に利用でき,かつ能力差があるグループでも,個々人が並行的に作業し,貢献できる構成管理ツールのモデルを考案し,実際の授業で試行した.システムの利用ログを利用することで初学者のグループプログラミングにおけるインタラクションを視覚化することで,インタラクションの実態解明と合わせてツールの評価を行った.

本論文は全\ref{CC}章からなる.プログラミング教育における「協調プログラミング」の定義を\ref{Def}章で行う.\ref{RW}章で,グループプログラミングを支援する既存ツールのレビューを行う.\ref{Model}章では,初学者に協調プログラミングを支援するツールのモデル「独立同期モデル」を提案する.\ref{CH}章では,\ref{Model}章で提案したモデルに基づいて設計・実装したシステムを説明する.\ref{EM}章で評価実験の方法,\ref{RS}章ではその結果を報告する.\ref{DC}章で考察を行う.\ref{CC}章はまとめである.

% pdfのテスト用のtexファイル
\chapter{pdfの挿入の仕方}
事前準備として,documentclassにdvipdfmxの追加と,usepackageにpdfpagesの追加が必要である.そして次のように記述することにより,文章中にpdfファイルを挿入することが可能である.中カッコ内はファイル名である.

\begin{verbatim}
\includepdf[pages=-, 
pagecommand={\thispagestyle{headings}, \markboth{}{}}, 
width=\textwidth, height=\textheight, frame=true]
{chapters/testpdfdata.pdf}
\end{verbatim}

文章にのせる都合上,複数行に分けたが,一形で書いても問題はない.


各オプションについて簡単に説明する.pagesのあとに来るものは,ページ数を表しているが,ここで-(ハイフンを)指定することにより,すべてのページを参照することが可能である.pagecommandの中では文章のスタイルを決めることができ,デフォルトではページ数が表示されない問題があったため,thispagestyleにてheadingを指定し,markbothにてページ数以外の情報を出力しないよう調整を行った.width,heightは挿入されるpdfのサイズを決めることができるが,デフォルトの設定ではページいっぱいいっぱいまで使ってしまうため,ここではtextwidth,textheightを用い,本文領域内に収まるように大きさを指定する.

その他のオプションについては下記サイトなどが参考になると思われる.

\begin{itemize}
\item pdfpages関係:http://abenori.blogspot.jp/2015/07/pdfpages.html
\item pdfpages詳細:http://texdoc.net/texmf-dist/doc/latex/pdfpages/pdfpages.pdf
\item thispagestyle関係:http://cns-guide.sfc.keio.ac.jp/2001/11/3/3.html
\item pageparameter関係:http://cns-guide.sfc.keio.ac.jp/2001/11/3/3.html
\end{itemize}

また,この方法で挿入を行うと,次ページ以降にpdfが挿入されることが確認されているので,よく注意していただきたい.

\includepdf[pages=-, 
pagecommand={\thispagestyle{headings}, \markboth{}{}}, 
width=\textwidth, height=\textheight, frame=true]
{chapters/testpdfdata.pdf}


%謝辞
%多分名前変えればそのまま使える
\chapter*{謝辞}

本研究の全過程を通じてご指導頂きました静岡大学大学院情報学研究科の情報太郎教授,ならびにhoge大学fuge学部のhogefuge助教に深く感謝するとともに,厚く御礼申し上げます.また,本研究のに対するご指導・ご助言を頂きました,静岡大学情報学部の教員の皆様に御礼申し上げます.そして,研究の期間中,ご協力頂きました酒井研究室の皆様,ならびに実験にご協力頂きました情報学部の学生に心から感謝の意を申し上げます.

%目次のページ数調整用
\newpage

%引数はbibファイル名(拡張子不要)
\reference{reference}

%著者のこれまでに発表した論文もあれば載せるらしい
\chapter*{著者発表論文}
\addcontentsline{toc}{chapter}{著者発表論文}
\begin{enumerate}
\renewcommand{\labelenumi}{[\arabic{enumi}]}

\item 静大花子, 情報太郎. 
プログラミング教育の研究.  
情報処理学会論文誌, vol.334, pp.334-352, 2014


\end{enumerate}

\end{document}